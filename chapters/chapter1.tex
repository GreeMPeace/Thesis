\chapter{Introduction}
In today's world, mobile communication has become extremely popular and highly essential to the social and economical part of our society. Nearly every person living in the industrial countries possesses at least one device capable of wireless communication of some sort. The high density of mobile devices in urban areas has lead to some interesting challenges in the planning and implementation of wireless network infrastructure. On the one hand, the network needs to be able to manage the high data throughput that is created by hundreds or thousands of devices. On the other hand, the network quality needs to be acceptable at any point in the area, because a high throughput is useless if the devices cannot access the network efficiently.

The network coverage problem is especially interesting in urban areas, because of the unique topology. Devices may be mounted atop a very tall skyscraper or far down in the streets, surrounded by buildings. Most of the commonly used wireless communication systems use electromagnetic waves to deliver information, which is no problem on a plane field. Within the rising and falling topology of an urban area however, the propagation of the network gets more difficult to predict. Just like it is with light, obstacles made of different material can absorb, reflect, alter or do nothing to any passing electromagnetic wave. This causes areas of high signal strength, where the antenna has a direct line of sight or a good reflecting path. However, it also crates areas of poor signal strength, for example behind a building (shadow) or in places, where reflections cause too much interference. Over time research institutions have addressed the problem and developed models to predict how network propagation behaves in obscured or confined areas. Many methods however are far to complex to be evaluated by hand for any realistic scenario. Hence, computer supported network simulation emerged. With the help of the computing power of modern computer programs like the WinProp Software Suite~\cite{WinProp} are able to forecast the behavior of wireless networks in different environments. Modern simulation applications can predict network propagation for large areas and many different antennas at the same time with decent accuracy. These tools provide a helpful overview over the network coverage, which is especially important for the current trend of infrastructure development.

Currently, the focus is shifting, when it comes to the design of wireless or radio networks in particular. Big macro cells, which provide efficient coverage in rural areas, often struggle in more urban areas, because of the way the buildings interact with the electromagnetic waves. Therefore big cell are being split up, to enhance throughput and directional preferences. Furthermore, newly installed cells are often micro cells, tailored to a specific location in performance and directional properties. This trend however makes it harder to plan where and how to install new infrastructure nodes, in order to satisfy demands. It takes careful analysis of wide range simulations and real life measurements to identify weak points in the coverage and patch them up or to increase signal quality in very demanding areas.

This is where good visualization applications come into play. There are already some tools (e.g. as part of WinProp~\cite{WinProp}) that try to help developers by visualizing the signal strength of simulated network scenarios. However, those mainly work in two dimensional space. From a top down perspective the network propagation is shown for a fixed height parameter. This can be helpful for engineers and developers. However, a realistic three dimensional representation of the environment with buildings, maps and the network propagation in between would be even more convenient. Any spacial arrangements and their problems would be easily viewable. For that reason, this thesis tries to first point out the challenges and prerequisites in creating a useful and user-friendly network visualization tool. Thereafter a possible implementation is presented. Finally, the application is analyzed, evaluated and expanded upon.