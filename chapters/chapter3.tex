\chapter{System Architecture}
\section{System Requirements}
\subsection{Hardware Requirements}
One of the most important hardware elements for the three dimensional visualization is the graphics~card\footnote{In some cases the hardware acceleration is disabled. See Section \ref{SoftReq} for further information.}. Depending on the size and level of detail of the scene there are a lot of computations that need to be done. Furthermore, the scene is supposed to be movable, so it cannot be a prerendered image. That makes some dedicated graphics hardware nearly indispensable. However, since the application renders simple polygon meshes and point clouds the graphics hardware does not have to be a high end device. It merely needs to relieve the CPU of some work. And with its processing unit made especially for matrix and vector calculation, any modern day graphics card that supports WebGL will meet the expectations.

CPU-only rendering is of course possible, too. The large amount of of points and surfaces in a complex scene however seriously slows down the rendering on an all purpose CPU. This also slows down the whole system, because of all the graphics calculations that are blocking the CPU. For a more detailed analysis see Section \ref{test}.

Another important hardware requirement is the RAM. Visualization data sets, especially from simulation applications, can get large very fast. City wide simulations with multiple antennas and a resolution of a few meters can easily have a few million data points. While the data that is being visualized might not take up much space as a file on the hard drive, within the application that might change. After being loaded and parsed, the data is stored within convenient data structures like arrays or objects. This leads to a less efficient compression of the data and also the addressing schemes and object methods add additional overhead. All that together results in an application that needs to load big chunks of data into the RAM. Therefor it is important that the system has enough memory at its disposal.
\subsection{Software Requirements} \label{SoftReq}
Since the main routine of the application, the rendering loop, runs in a web browser as JavaScript code, it is important to have a browser, that supports all the used functionality. Firstly, and most importantly, the renderer used here is a so called ''webGL-renderer´´, so it is important that the browser even supports WebGL. All current versions of the commonly used ones (e.g. Chrome, Firefox, Safari, \ldots) are able to support HTML5, CSS3 and WebGL. It is advisable to use the most recent update of the browser, because the developers always improve the performance or fix bugs. Also the support of WebGL grew over time, so very old versions may not support all the features used in this application. For lesser known browsers the functionality has to be determined individually. Important criteria are the former mentioned HTML5, CSS3 and WebGL support. It is also worth mentioning that the browser needs to allow WebGL to use hardware accelerated rendering, in other word access the graphics card. Microsoft's Internet Explorer for example does not do that. This leads to the problem, that it uses CPU-only rendering even if a graphics card is installed, which in turn, slows down the whole system.
\section{Architectural Design}
\subsection{Initial JavaScript code}
\subsection{Frontend -- Backend}