\usepackage[latin1]{inputenc} % latin1 for Windows and older Linuxes,
                              % utf8 for newer Linuxes
\usepackage[T1]{fontenc}


%\usepackage{marvosym} % Martin Vogel's symbol package
%\usepackage{rotating} % Rotate things as you like
%\usepackage{lscape}   % Rotate text to landscape
%\usepackage{fancyvrb} % Fancy package for reading and writing verbatim TeX code

% ... and environments
\renewenvironment{quote}%
   {\begin{quotation}\noindent\itshape}%
   {\upshape\end{quotation}}

% ----- Hyphenation! -----
% Now special cases could be defined and how these should always be hyphenated.
\hyphenation{ }

\usepackage{listings}

\usepackage{color}
\definecolor{lightgray}{rgb}{.95,.95,.95}
\definecolor{darkgray}{rgb}{.3,.3,.3}
\definecolor{lightgreen}{rgb}{.2823529411764706,.7333333333333333,0}
\definecolor{purple}{rgb}{0.65,0.12,0.82}
\lstdefinelanguage{JavaScript}{
  keywords=[1]{},
  keywords=[2]{break, case, catch, continue, debugger, default, delete, do, else, false, finally, for, function, if, in, instanceof, new, null, return, switch, true, try, typeof, var, void, while, with},
  keywordsprefix=0x,
  morecomment=[l]{//},
  morecomment=[s]{/*}{*/},
  morestring=[b]',
  morestring=[b]",
  keywords=[3]{class, export, boolean, throw, implements, import, this},
  keywordstyle=\color{black},
  keywordstyle=[2]\color{lightgreen}\bfseries,
  keywordstyle=[3]\color{lightgreen}\bfseries,
  identifierstyle=\color{black}\bfseries,
  commentstyle=\color{darkgray}\ttfamily,
  stringstyle=\color{red}\ttfamily,
  sensitive=true
}

\lstset{
   language=JavaScript,
   backgroundcolor=\color{lightgray},
   extendedchars=true,
   basicstyle=\footnotesize\ttfamily,
   showstringspaces=false,
   showspaces=false,
   tabsize=2,
   breaklines=true,
   showtabs=false,
   captionpos=b
}